\chapter{Apresentação teórica} \label{chap:apresentacao_teorica}

Neste capítulo, alguns conceitos teóricos são explanados com o intuito de fornecer ao leitor o embasamento teórico necessário para a compreensão completa deste trabalho.

\section{Aprendizado de máquina}

O campo de estudo de aprendizado de máquina é uma vasta área da computação, possuindo várias aplicações, objetos de estudo e focos de pesquisa interdisciplinares. Resumidamente, podemos dizer que essa área estuda como "construir programas de computador que melhoram seu desempenho em alguma tarefa por meio da experiência"\footnote{\cite[p.29]{machine_learning}}. Atualmente, utilizamos o aprendizado de máquina para várias aplicações como algoritmos de recomendações de conteúdo, programas de reconhecimento e classificação de imagens e a realização de análises de risco financeiro.

Para utilizarmos o aprendizado de máquina para resolvermos alguma problema, faz-se necessário definir matematicamente uma tarefa a ser realizada, uma ou mais métricas de desempenho atreladas à realização da tarefa e a fonte de experiência que será utilizada pelo modelo para aprender a realizar a tarefa.\footnote{\cite[p.29]{machine_learning}} Também precisamos escolher um ou mais tipos de modelo de aprendizado de máquina que serão utilizados para aprender a realizar a tarefa com base em um treinamento feito com a fonte de experiência avaliado sob a ótica das métricas de desempenho escolhidas.

\subsection{Especificação de uma tarefa}

Qualquer problema de aprendizado de máquina deve possuir uma tarefa a ser realizada, que representa o objetivo a ser atingido pelo uso de aprendizado de máquina. A especificação dessa tarefa sempre deve possuir o formato de uma função matemática para permitir que um programa de computador consiga aprendê-la. Assim, podemos afirmar que qualquer tarefa de aprendizado de máquina pode ser representada, genericamente, pela função $T : V \rightarrow R$, onde $V$ é um conjunto de variáveis disponibilizado para a realização da tarefa de aprendizado de máquina e $R$ é o resultado esperado da tarefa de aprendizado de máquina.

Para um programa de aprendizado de máquina realizar a tarefa proposta, este deve aprender a função $T$. Entretanto, na grande maioria dos problemas de aprendizado de máquina, a função $T$ não é conhecida e o problema proposto se resume a aproximar uma \textit{descrição operacional} de $T$.\footnote{\cite[p.8]{machine_learning}} Dessa forma, o programa de aprendizado de máquina deve então utilizar o processo de aprendizado com base na fonte de experiência para adquirir uma função $\hat{T}$ que seja uma boa aproximação da função $T$.

Neste trabalho, o problema de previsão de insuficiência cardíaca pode ser descrito como sendo um problema de \textit{classificação binária}, em que a tarefa proposta é representada pela função $P : V_{p} \rightarrow [0, 1]$, onde $V_{p}$ são as variáveis fornecidas que descrevem o paciente e o conjunto imagem $[0, 1]$ representa uma previsão se o paciente irá vir a óbito (1) ou não (0).

\subsection{Fonte de experiência}

Para permitir que um programa de aprendizado de máquina aprenda uma aproximação $\hat{T}$ da função $T$, é necessário a utilização de uma fonte de experiência que forneça, direta ou indiretamente, uma maneira do programa de treinamento inferir o comportamento da função $T$. Essa fonte de experiência pode ser obtida de várias formas, as quais variam consideravelmente dependendo da tarefa em questão e do método de aprendizado almejado para o programa computacional, de forma que o tipo de recurso utilizado como fonte de experiência não apenas é determinante no sucesso ou fracasso do aprendizado, como também define uma categoria de aprendizado que será adotada pelo programa computacional.

O tipo mais comum de fonte de experiência utilizada é um conjunto de dados com exemplos que possuem variáveis e o resultado da aplicação dessas variáveis à função $T$, categoria de aprendizado conhecida como aprendizado supervisionado. Alguns outros tipos de fontes de experiência e suas respectivas categorias de aprendizado incluem: o uso de um conjunto de dados com exemplos com variáveis mas sem nenhum resultado da função $T$, categoria conhecida como aprendizado não supervisionado; o uso de um conjunto de dados com exemplos com variáveis mas nem sempre com o resultado da aplicação dessas variáveis à função $T$, categoria conhecida como aprendizado semi-supervisionado; e uma exploração da função $T$ feita pelo próprio programa computacional com base em uma métrica de recompensa, categoria conhecida como aprendizado por reforço.

Para que a função de aproximação $\hat{T}$ aprendida pelo programa de aprendizado de máquina com base na fonte de experiência utilizada seja uma boa aproximação da função $T$, é necessário que a fonte de experiência utilizada seja uma boa aproximação dos exemplos que o programa de aprendizado de máquina encontrará ao longo de sua avaliação e uso, uma suposição que quase sempre se revela como falsa.\footnote{\cite[p.6]{machine_learning}} Isso pode levar a ocorrência de um fenômeno denominado sobreajuste (ou \textit{overfitting}) no processo de aprendizado do programa computacional. Esse fênomeno ocorre quando o processo de aprendizado é aplicado excessivamente na fonte de experiência utilizada, piorando a aproximação $\hat{T}$ aprendida pelo programa computacional devido às discrepâncias entre os dados englobados na fonte de experiência e os dados englobados em todo o domínio da função $T$.

Neste trabalho, o programa de previsão de insuficiência cardíaca utilizou como fonte de experiência para seu treinamento um conjunto de dados\footnote{\cite{larxel_dataset}} composto de exemplos com variáveis e o resultado da aplicação dessas variáveis à função $T$, tomando parte, assim, na categoria de problemas de aprendizado supervisionado. O capítulo \ref{chap:procedimento_adotado} aborda alguns cuidados utilizados para evitar a ocorrência de sobreajuste no processo de aprendizado.

