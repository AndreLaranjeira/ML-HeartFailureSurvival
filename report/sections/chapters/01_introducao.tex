\chapter{Introdução} \label{chap:introducao}

Este trabalho realiza um estudo comparativo de modelos de aprendizado de máquina do tipo perceptron multicamada e floresta aleatória treinados com um conjunto de variáveis relacionadas à saúde cardiovascular e geral para realizar uma previsão da sobrevivência à insuficiência cardíaca de um paciente humano durante um período de acompanhamento médico de duração média de 130 dias. Além disso, neste trabalho também foi implementado um website para permitir que o modelo de treinamento com a melhor acurácia de previsão no estudo mencionado possa ser acessado e utilizado de forma simples e direta por um público alvo abrangente.

Com este trabalho, espera-se exemplificar como uma aplicação concreta, mais especificamente um website, pode ser construída ao redor de um estudo de aprendizado de máquina para permitir que um modelo de aprendizado de máquina seja utilizado por profissionais da saúde para fornecer benefícios concretos no atendimento a pacientes.

O trabalho está organizado no formato descrito a seguir. O capítulo \ref{chap:apresentacao_teorica} apresenta ao leitor conceitos teóricos fundamentais para o entendimento deste trabalho. O capítulo \ref{chap:procedimento_adotado} descreve o procedimento adotado pelo autor na realização deste trabalho. O capítulo \ref{chap:resultados} apresenta os resultados obtidos neste trabalho. Por fim, o capítulo \ref{chap:conclusao} apresenta a conclusão deste trabalho e sugere futuras pesquisas a serem realizadas com base neste trabalho.

Todo o código fonte utilizado para este trabalho pode ser encontrado no repositório do \textit{GitHub} para este trabalho\footnote{\url{https://github.com/AndreLaranjeira/ML-HeartFailureSurvival}}.

\section{Relevância da temática}

A insuficiência cardíaca é uma condição clínica crônica e progressiva em que o coração não consegue bombear sangue suficiente para todo o corpo resultando em sintomas como fadiga e problemas respiratórios \cite{heart_failure_definition}. Os principais fatores que aumentam o risco de desenvolvimento de insuficiência cardíaca são doenças arteriais coronarianas, hipertensão, diabetes, obesidade e fumo \cite[p.399]{heart_disease2021}. Além disso, estudos recentes demonstraram que a falta de boas condições físicas nos sistemas cardíaco e respiratório também contribui para o aumento do risco de desenvolvimento de insuficiência cardíaca \cite[p.62]{heart_disease2021}.

A ocorrência de insuficiência cardíaca possui grande relevância nos dias atuais. Segundo a Associação Americana do Coração (AHA)\nomenclature{AHA}{Associação Americana do Coração}, dados coletados entre 2015 e 2018 apontam que 6 milhões de adultos estadunidenses com mais de 20 anos tiveram insuficiência cardíaca \cite[p.8]{heart_disease2021} e que 83.616 estadunidenses vieram a óbito em 2018 por insuficiência cardíaca \cite[p.485]{heart_disease2021}. No Brasil, um estudo publicado em 2019 na revista Acta Fisiátrica \cite{nogueira2019} utilizou dados de 2013 para estimar que 1,7 milhões de brasileiros já tiveram insuficiência cardíaca e outro estudo publicado em 2014 nos Arquivos Brasileiros de Cardiologia \cite{gaui2014} aponta que aproximadamente 27.702 brasileiros vieram a óbito em 2011 por insuficiência cardíaca.

Por fim, devemos salientar o impacto econômico causado por essa condição clínica. De acordo com uma diretriz de insuficiência cardíaca dos Arquivos Brasileiros de Cardiologia \cite{diretriz_insuficiencia_cardiaca_2018}, os gastos globais governamentais e privados com essa condição clínica foram de R\$ 14,5 bilhões apenas em 2015.

O uso de programas de aprendizado de máquina que prevejam a ocorrência de insuficiência cardíaca e a mortalidade por insuficiência cardíaca pode auxiliar na redução do número de pessoas que sejam afligidas e que venham a óbito por essa condição médica, respectivamente. Isso resultaria na diminuição dos recursos monetários despendidos globalmente, direta ou indiretamente, no tratamento da insuficiência cardíaca. Além disso, a saúde geral da população mundial seria beneficiada com a diminuição do número de ocorrências de insuficiência cardíaca.

\section{Literatura existente}

Artigos recentes exploraram vários aspectos sobre como o aprendizado de máquina pode ser utilizado para combater a insuficiência cardíaca. Um artigo escrito por Awan, Sohel e outros revisou estudos existentes acerca das possibilidades de uso de aprendizado de máquina para melhorar os tratamentos e diagnósticos de insuficiência cardíaca e assim diminuir gastos com essa condição médica \cite{awan2018}. Um outro artigo publicado no jornal europeu de insuficiência cardíaca utiliza o aprendizado de máquina para correlacionar pacientes e desenvolver uma nova métrica de risco de óbito por insuficiência cardíaca que obteve mais acurácia que as métricas existentes \cite{adler2020}. Por fim, um outro artigo publicado no jornal da sociedade americana de ecocardiografia propós um programa de aprendizado de máquina para identificar a presença de insuficiência cardíaca com preservação na taxa de ejeção, um tipo específico de insuficiência cardíaca, por meio de imagens ecocardiográficas que mensurem as velocidades miocárdicas em repouso e durante exercício \cite{hfpef2018}.

Dentre a literatura existente, forneceremos destaque especial ao artigo publicado por Chicco e Jurman \cite{chicco2020}. Esse artigo faz uso de um conjunto de dados de pacientes que foram hospitalizados com insuficiência cardíaca \cite{larxel_dataset} disponibilizado em um artigo científico de 2017 \cite{dataset_article}, utilizando esse conjunto de dados tanto para realizar uma análise comparativa da performance obtida na previsão de sobrevivência à insuficiência cardíaca com o uso de difentes modelos de treinamento e variáveis dos pacientes como para identificar quais variáveis do conjunto de dados possuem maior correlação com a ocorrência de óbito por insuficiência cardíaca. Adotamos o artigo publicado por Chicco e Jurman como a principal referência utilizada neste trabalho devido à semelhança dos objetivos propostos para o uso do aprendizado de máquina neste trabalho e no artigo em questão e ao fato de que este trabalho utiliza o mesmo conjunto de dados que o artigo em questão como fonte de experiência do aprendizado de máquina.
