\chapter{Introdução} \label{chap:introducao}

Este trabalho realiza um estudo comparativo de modelos de aprendizado de máquina do tipo perceptron multicamada e floresta aleatória treinados para realizar a previsão da ocorrência de insuficiência cardíaca em um paciente humano com base em um conjunto de variáveis representativo da saúde cardiovascular e geral do paciente.

Além disso, neste trabalho também foi implementado um website para permitir que o modelo de treinamento com a melhor acurácia de previsão no estudo mencionado possa ser acessado e utilizado de forma simples e direta por um público alvo abrangente. Por meio desse website, espera-se que profissionais de saúde possam obter algum valor material concreto do modelo de aprendizado treinado com o propósito de prever a ocorrência de insuficiência cardíaca em um paciente humano.

O trabalho está organizado no formato descrito a seguir. O capítulo \ref{chap:apresentacao_teorica} apresenta ao leitor conceitos teóricos fundamentais para o entendimento deste trabalho. O capítulo \ref{chap:procedimento_adotado} descreve o procedimento adotado pelo autor na realização deste trabalho. O capítulo \ref{chap:resultados} apresenta os resultados obtidos neste trabalho. Por fim, o capítulo \ref{chap:conclusao} apresenta a conclusão deste trabalho e sugere futuras pesquisas a serem realizadas com base neste trabalho.

Todo o código fonte utilizado para este trabalho pode ser encontrado no repositório do \textit{GitHub} para este trabalho\footnote{\url{https://github.com/AndreLaranjeira/ML-HeartFailure}}.

\section{Literatura existente}

A ocorrência de insuficiência cardíaca como resultado de doenças cardiovasculares possui grande relevância nos dias atuais. Segundo a American Heart Association (AHA)\nomenclature{AHA}{American Heart Association}, dados coletados entre 2015 e 2018 apontam que 6 milhões de adultos estadunidenses com mais de 20 anos tiveram insuficiência cardíaca\cite[p.8]{heart_disease2021}. A AHA também estima que, apenas nos Estados Unidos da América, a média dos custos diretos e indiretos associados à categoria de doenças cardíacas foi estimado em 219,6 bilhões de dólares no ano fiscal estadunidense de 2016/2017\cite[p.481]{heart_disease2021}. A categoria mencionada engloba os gastos com insuficiência cardíaca e outras doenças cardíacas, mas não engloba os gastos com hipertensão e enfarto, os quais possuem categorias separadas na análise feita pela AHA.

Nesse contexto, o uso de programas de aprendizado de máquina que prevejam a ocorrência de insuficiência cardíaca pode auxiliar na redução do número de pessoas que sejam afligidas por essa condição médica. Isso resultaria na diminuição dos recursos monetários despendidos globalmente, direta ou indiretamente, no tratamento da insuficiência cardíaca. Além disso, a saúde geral da população mundial seria beneficiada com a diminuição do número de ocorrências de insuficiência cardíaca.

Artigos recentes exploraram vários aspectos sobre como o aprendizado de máquina pode ser utilizado para combater a insuficiência cardíaca. Um artigo escrito por Awan, Sohel e outros revisou estudos existentes acerca das possibilidades de uso de aprendizado de máquina para melhorar os tratamentos e diagnósticos de insuficiência cardíaca e assim diminuir gastos com essa condição médica\cite{awan2018}. Um outro artigo publicado no jornal europeu de insuficiência cardíaca utiliza o aprendizado de máquina para correlacionar pacientes e desenvolver uma nova métrica de risco de óbito por insuficiência cardíaca que obteve mais acurácia que as métricas existentes\cite{adler2020}. Por fim, um outro artigo publicado no jornal da sociedade americana de ecocardiografia propós um programa de aprendizado de máquina para identificar a presença de insuficiência cardíaca com preservação na taxa de ejeção, um tipo específico de insuficiência cardíaca, por meio de imagens ecocardiográficas que mensurem as velocidades miocárdicas em repouso e durante exercício\cite{hfpef2018}.

Dentre a literatura existente, forneceremos destaque especial ao artigo publicado por Chicco e Jurman\cite{chicco2020}. Esse artigo disponibiliza um conjunto de dados de pacientes que foram hospitalizados com insuficiência cardíaca\cite{larxel_dataset} e utiliza esse conjunto de dados tanto para realizar uma análise comparativa da performance obtida na previsão de óbito por insuficiência cardíaca com o uso de difentes modelos de treinamento e variáveis dos pacientes como para identificar quais variáveis do conjunto de dados possuem maior correlação com a ocorrência de óbito por insuficiência cardíaca. Adotamos esse artigo como a principal referência utilizada neste trabalho devido à semelhança dos objetivos propostos para o uso do aprendizado de máquina neste trabalho e no artigo em questão e ao fato de que este trabalho utiliza o conjunto de dados disponibilizado pelo artigo em questão como fonte de experiência do aprendizado de máquina.
