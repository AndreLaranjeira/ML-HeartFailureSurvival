\chapter{Introdução} \label{chap:introducao}

Este trabalho tem o intuito de realizar um estudo comparativo de modelos de aprendizado de máquina do tipo perceptron multicamada e floresta aleatória treinados para realizar a previsão da ocorrência de insuficiência cardíaca em um paciente humano com base em um conjunto de variáveis representativo da saúde cardiovascular e geral do paciente.

Além disso, este trabalho também busca implementar um website para permitir que o modelo de treinamento com a melhor acurácia de previsão no estudo mencionado possa ser acessado e utilizado de forma simples e direta por um público alvo abrangente. Por meio desse website, espera-se que os usuários possam obter algum valor material concreto do modelo de aprendizado treinado com o propósito de prever a ocorrência de insuficiência cardíaca em um paciente humano.

O trabalho está organizado no formato descrito a seguir. O capítulo \ref{chap:apresentacao_teorica} apresenta ao leitor conceitos teóricos fundamentais para o entendimento deste trabalho. O capítulo \ref{chap:procedimento_adotado} descreve o procedimento adotado pelo autor na realização deste trabalho. O capítulo \ref{chap:resultados} apresenta os resultados obtidos neste trabalho. Por fim, o capítulo \ref{chap:conclusao} apresenta a conclusão deste trabalho e sugere futuras pesquisas a serem realizadas com base neste trabalho.

Todo o código fonte utilizado para este trabalho pode ser encontrado no repositório do \textit{GitHub} para este trabalho\footnote{\url{https://github.com/AndreLaranjeira/ML-HeartFailure}}.
