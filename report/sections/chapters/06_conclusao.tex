\chapter{Conclusão} \label{chap:conclusao}

Com base na introdução feita no capítulo \ref{chap:introducao} acerca dos objetivos deste trabalho e com base nos resultados deste trabalho apresentados no capítulo \ref{chap:resultados}, podemos concluir que este trabalho atingiu o objetivo proposto.

O estudo comparativo de modelos do tipo perceptron multicamada e florestas randômicas descrito nos capítulos \ref{chap:procedimento_adotado} e \ref{chap:resultados} foi capaz de enunciar claras distinções na acurácia de previsão do modelo de treinamento geradas pelo tipo de modelo de aprendizado de máquina utilizado e pelos hiperparâmetros escolhidos. A melhor média de acurácia de previsão obtida sobre os 100 subconjuntos de teste por um modelo do tipo florestas randômicas com base em um treinamento feito sobre 10 das 12 variáveis do conjunto de dados de insuficiência cardíaca utilizado\footnote{\cite{larxel_dataset}} foi de 76\%, superando a melhor marca de 74\% registrada no artigo de Chicco e Jurman\footnote{\cite{chicco2020}} para modelos do tipo florestas randômicas treinados com todas as 12 variáveis do conjunto de dados.

O website implementado funcionou adequadamente, conforme mencionado no capítulo \ref{chap:resultados}. Embora uma aplicação que oferece uma previsão de insuficiência cardíaca com acurácia de 76\% não possa ser disponibilizada ao público por razões éticas e legais relacionadas ao campo da medicina, podemos afirmar que o website implementado fornece um valor concreto aos seus usuários ao permitir que eles organizem as informações de seus pacientes e requisitem previsões de insuficiência cardíaca para estes.
