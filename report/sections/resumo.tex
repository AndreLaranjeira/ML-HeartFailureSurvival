\chapter*{Resumo}

Este trabalho consiste em uma comparação entre modelos de aprendizado de máquina do tipo perceptron multicamada e floresta aleatória treinados para a previsão de insuficiência cardíaca e em um website auxiliar para utilização do melhor modelo. A avaliação dos modelos de treinamento se baseou na melhor média de acurácia de previsão envolvendo 20 subconjuntos de validação e 100 subconjuntos de teste. Ao total 5346 modelos de treinamento foram avaliados e o modelo mais bem classificado obteve uma média de acurácia comparável àquela do artigo de referência utilizado. A implementação do website auxiliar também obteve êxito ao simplificar o acesso ao melhor modelo de previsão. Para trabalho futuros, planeja-se a avaliação de mais tipos de modelo de aprendizado de máquina e suas combinações de hiperparâmetros e a realização de testes do melhor modelo com pacientes contemporâneos.
