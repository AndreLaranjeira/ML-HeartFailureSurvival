% Classe de documento.
\documentclass[a4paper,12pt,openright,titlepage,oneside]{book}

%Define regra de gramática para separar síbalas {babel}
%e altera os títulos (como chapters, sections, references) para português
\usepackage[brazil]{babel}

% Define uso de caracteres acentuados no PDF gerado
% e permite copiar corretamente o texto do PDF.
\usepackage[T1]{fontenc}
\usepackage[utf8]{inputenc}

% Outros pacotes utilizados.
\usepackage{subfigure}

% Adição do pacote do template da UnB.
\usepackage{styles/template-FT-UnB/ft2unb}

\DeclareGraphicsExtensions{.jpg,.pdf,.mps,.png,.gif, .eps}
\graphicspath{images} %define diretório de imagens

%Arquivo com lista com hifenização correta de algumas palavras.
%Defina novas palavras no arquivo a medida que verificar que a hifenização automática
%etá errada para tais palavras.
\input{styles/hifenizacao}

% ALTERE OS VALORES DENTRO DAS CHAVES DOS COMANDOS NESTA SEÇÃO PARA INCLUIR OS SEUS DADOS E DADOS DA SUA
% DISSERTAÇÃO DE MESTRADO OU TESE DE DOUTORADO
% -----------------------------------------------------------------------------------------------------

%\onehalfspacing
\title{Previsão de insuficiência cardíaca com aprendizado de máquinas em um website}
\author{André Filipe Caldas Laranjeira}
\date{2021-05-21} %data da defesa
\coverbackgroundimg{images/capa_fundo}

\grau{Bacharel}
\area{Engenharia da computação} %Nome do curso
\departamento{Ciência da computação} %Nome do departamento
\faculdade{INSTITUTO DE CIÊNCIAS EXATAS} %Nome da faculdade/instituto
\siglafaculdade{IE} %Sigla da faculdade/instituto
\siglaarea{CIC} %Sigla do departamento
\tipodemonografia{Monografia} %Dissertação ou Tese
\programa{Bacharelado} %Mestrado ou Doutorado
\autorendereco{Área Octogonal Sul 2, Bloco G, Apartamento 601; Octogonal, Brasília} %Endereço do autor da dissertação/tese
\totalpgs{32} %total de páginas atualmente na sua dissertação
\dia{21} %dia da defesa
\mes{Maio} %mês da defesa
\ano{2021} %ano da defesa
\numpublicacao{xxx/AAAA} %número da publicação (após a defesa, tal número deve ser obtido na secretaria)

%PPGENE.DM  = Programa de Pós Graduação em ENgenharia Elétrica.Dissertação de Mestrado
%PPGENE.TD  = Programa de Pós Graduação em ENgenharia Elétrica.Tese de Doutorado
\siglapublicacao{?}

\titulolinhai{Previsão de insuficiência cardíaca}
\titulolinhaii{com aprendizado de máquinas}
\titulolinhaiii{em um website}

\autori{André Filipe Caldas Laranjeira}
%Caso seu nome não caiba em uma única linha, divida ele nos comandos abaixo
%\autorii{}
%\autoriii{}

\membrodabancai{Prof. Dr. Alexandre Ricardo Soares Romariz, FT/UnB}
\membrodabancaifuncao{Orientador}
\membrodabancaii{Prof. Fulano de Tal 2, ENE/UnB}
\membrodabancaiifuncao{Examinador interno}
\membrodabancaiii{Prof. Fulano de Tal 3, ENE/UnB}
\membrodabancaiiifuncao{Examinador interno}
% -----------------------------------------------------------------------------------------------------

%line-numbers, inputencoding=utf8/latin1
%Define o estilo para listagens de código fonte
\lstset{
  numbers=left, %numeração de linhas à esquerda
  stepnumber=1,
  firstnumber=1,
  numberstyle=\tiny,
  extendedchars=true,
  frame=none,
  basicstyle=\footnotesize,
  stringstyle=\ttfamily,
  showstringspaces=false,
  %language=Java, %deve ser definida na inclusão de cada trecho de código, pois podem existir linguagens diferentes em exemplos diferentes
  breaklines=true,
  breakautoindent=true,
  %estilos de comentário de uma e várias linhas
  morecomment=[l]{--}, morecomment=[s]{/*}{*/}, morecomment=[s]{<!--}{-->}, morecomment=[s]{--[[}{--]]}
}

% Adição de metadados no PDF (propriedades do documento PDF)
\makeatletter
	 \hypersetup{
		 pdftoolbar=true,        % show Acrobat’s toolbar?
		 pdfmenubar=true,        % show Acrobat’s menu?
		 pdffitwindow=false,     % window fit to page when opened
		 pdfstartview={FitH},    % fits the width of the page to the window
		 pdftitle={\@title},
		 pdfauthor={\@author},
		 pdfsubject={\tipodemonografianome \ de\ \programastr \ em\ \areastr},   % subject of the document
		 pdfcreationdate={\pdfdate}
	 }
\makeatother


\makeindex
\makenomenclature %Necessário para gerar lista de siglas


\begin{document}

	\pdfbookmark[0]{Agradecimentos}{agradecimentos}
	\chapter*{Agradecimentos}

Agradeço a Deus por ter me abençoado e me guiado durante todo o meu curso, cuidando de minha vida de maneira maravilhosa.
Agradeço à minha família por ter me sustentado até aqui e me proporcionado amor e encorajamento.
Agradeço aos meus professores por tudo o que eles me ensinaram e pelos desafios que eles me propuseram.
Agradeço aos meus colegas de curso por terem me ajudado ao longo do curso e me inspirado a sempre dar o melhor de mim.

	\chapter*{Resumo}

Este trabalho consiste em uma comparação entre modelos de aprendizado de máquina do tipo perceptron multicamada e floresta aleatória treinados para a previsão de insuficiência cardíaca e em um website auxiliar para utilização do melhor modelo. A avaliação dos modelos de treinamento se baseou na melhor média de acurácia de previsão envolvendo 20 subconjuntos de validação e 100 subconjuntos de teste. Ao total 5346 modelos de treinamento foram avaliados e o modelo mais bem classificado obteve uma média de acurácia comparável àquela do artigo de referência utilizado. A implementação do website auxiliar também obteve êxito ao simplificar o acesso ao melhor modelo de previsão. Para trabalho futuros, planeja-se a avaliação de mais tipos de modelo de aprendizado de máquina e suas combinações de hiperparâmetros e a realização de testes do melhor modelo com pacientes contemporâneos.


	\pdfbookmark[0]{Sumário}{sumario}
	\sumario

	\pdfbookmark[0]{Lista de Figuras}{listafiguras}
	\listadefiguras

	\pdfbookmark[0]{Lista de Tabelas}{listatabelas}
	\listadetabelas

	\pdfbookmark[0]{Lista de Códigos Fonte}{listacodigosfonte}
	\listadecodigosfonte

	\renewcommand{\nomname}{LISTA DE TERMOS E SIGLAS} %Define um caption à lista de siglas
	%Inclui a lista de siglas
	\pdfbookmark[0]{Lista de Termos e Siglas}{nomenclatura}
	\printnomenclature[2.5cm]

	\mainmatter %Inicia a numeracao normal cardinal
	\setcounter{page}{1} \pagenumbering{arabic} \pagestyle{plain}

	\chapter{Introdução} \label{chap:introducao}

Este trabalho tem o intuito de realizar um estudo comparativo de modelos de aprendizado de máquina do tipo perceptron multicamada e floresta aleatória treinados para realizar a previsão da ocorrência de insuficiência cardíaca em um paciente humano com base em um conjunto de variáveis representativo da saúde cardiovascular e geral do paciente.

Além disso, este trabalho também busca implementar um website para permitir que o modelo de treinamento com a melhor acurácia de previsão no estudo mencionado possa ser acessado e utilizado de forma simples e direta por um público alvo abrangente. Por meio desse website, espera-se que os usuários possam obter algum valor material concreto do modelo de aprendizado treinado com o propósito de prever a ocorrência de insuficiência cardíaca em um paciente humano.

O trabalho está organizado no formato descrito a seguir. O capítulo \ref{chap:apresentacao_teorica} apresenta ao leitor conceitos teóricos fundamentais para o entendimento deste trabalho. O capítulo \ref{chap:procedimento_adotado} descreve o procedimento adotado pelo autor na realização deste trabalho. O capítulo \ref{chap:resultados} apresenta os resultados obtidos neste trabalho. Por fim, o capítulo \ref{chap:conclusao} apresenta a conclusão deste trabalho e sugere futuras pesquisas a serem realizadas com base neste trabalho.

Todo o código fonte utilizado para este trabalho pode ser encontrado no repositório do \textit{GitHub} para este trabalho\footnote{\url{https://github.com/AndreLaranjeira/ML-HeartFailure}}.

	\chapter{Apresentação teórica} \label{chap:apresentacao_teorica}

Neste capítulo, alguns conceitos teóricos são explanados com o intuito de fornecer ao leitor o embasamento teórico necessário para a compreensão completa deste trabalho.

\section{Aprendizado de máquina}

O campo de estudo de aprendizado de máquina é uma vasta área da computação, possuindo várias aplicações, objetos de estudo e focos de pesquisa interdisciplinares. Resumidamente, podemos dizer que essa área estuda como "construir programas de computador que melhoram seu desempenho em alguma tarefa por meio da experiência"\cite[p.29]{machine_learning}. Atualmente, utilizamos o aprendizado de máquina para várias aplicações como algoritmos de recomendações de conteúdo, programas de reconhecimento e classificação de imagens e a realização de análises de risco financeiro.

Para utilizarmos o aprendizado de máquina para resolvermos alguma problema, faz-se necessário definir matematicamente uma tarefa a ser realizada, uma ou mais métricas de desempenho atreladas à realização da tarefa e a fonte de experiência que será utilizada pelo modelo para aprender a realizar a tarefa\cite[p.29]{machine_learning}. Também precisamos escolher um ou mais tipos de modelo de aprendizado de máquina que serão utilizados para aprender a realizar a tarefa com base em um treinamento feito com a fonte de experiência avaliado sob a ótica das métricas de desempenho escolhidas.

\subsection{Especificação de uma tarefa}

Qualquer problema de aprendizado de máquina deve possuir uma tarefa a ser realizada, que representa o objetivo a ser atingido pelo uso de aprendizado de máquina. A especificação dessa tarefa sempre deve possuir o formato de uma função matemática para permitir que um programa de computador consiga aprendê-la. Assim, podemos afirmar que qualquer tarefa de aprendizado de máquina pode ser representada, genericamente, pela função $T : V \rightarrow R$, onde $V$ é um conjunto de variáveis disponibilizado para a realização da tarefa de aprendizado de máquina e $R$ é o resultado esperado da tarefa de aprendizado de máquina.

Para um programa de aprendizado de máquina realizar a tarefa proposta, este deve aprender a função $T$. Entretanto, na grande maioria dos problemas de aprendizado de máquina, a função $T$ não é conhecida e o problema proposto se resume a aproximar uma \textit{descrição operacional} de $T$\cite[p.8]{machine_learning}. Dessa forma, o programa de aprendizado de máquina deve então utilizar o processo de aprendizado com base na fonte de experiência para adquirir uma função $\hat{T}$ que seja uma boa aproximação da função $T$.

Neste trabalho, o problema de previsão de insuficiência cardíaca pode ser descrito como sendo um problema de \textit{classificação binária}, em que a tarefa proposta é representada pela função $P : V_{p} \rightarrow \{0, 1\}$, onde $V_{p}$ são as variáveis fornecidas que descrevem o paciente e o conjunto imagem $\{0, 1\}$ representa uma previsão se o paciente irá sofrer complicações devido à insuficiência cardíaca (1) ou não (0).

\subsection{Fonte de experiência}

Para permitir que um programa de aprendizado de máquina aprenda uma aproximação $\hat{T}$ da função $T$, é necessário a utilização de uma fonte de experiência que forneça, direta ou indiretamente, uma maneira do programa de treinamento inferir o comportamento da função $T$. Essa fonte de experiência pode ser obtida de várias formas, as quais variam consideravelmente dependendo da tarefa em questão e do método de aprendizado almejado para o programa computacional, de forma que o tipo de recurso utilizado como fonte de experiência não apenas é determinante no sucesso ou fracasso do aprendizado, como também define uma categoria de aprendizado que será adotada pelo programa computacional.

O tipo mais comum de fonte de experiência utilizada é um conjunto de dados com exemplos que possuem variáveis e o resultado da aplicação dessas variáveis à função $T$, categoria de aprendizado conhecida como aprendizado supervisionado. Alguns outros tipos de fontes de experiência e suas respectivas categorias de aprendizado incluem: o uso de um conjunto de dados com exemplos com variáveis mas sem nenhum resultado da função $T$, categoria conhecida como aprendizado não supervisionado; o uso de um conjunto de dados com exemplos com variáveis mas nem sempre com o resultado da aplicação dessas variáveis à função $T$, categoria conhecida como aprendizado semi-supervisionado; e uma exploração da função $T$ feita pelo próprio programa computacional com base em uma métrica de recompensa, categoria conhecida como aprendizado por reforço.

Para que a função de aproximação $\hat{T}$ aprendida pelo programa de aprendizado de máquina com base na fonte de experiência utilizada seja uma boa aproximação da função $T$, é necessário que a fonte de experiência utilizada seja uma boa aproximação dos exemplos que o programa de aprendizado de máquina encontrará ao longo de sua avaliação e uso, uma suposição que é apenas parcialmente verdadeira\cite[p.6]{machine_learning}. Isso pode levar a ocorrência de um fenômeno denominado sobreajuste (ou \textit{overfitting}) no processo de aprendizado do programa computacional. O sobreajuste ocorre quando o uso de uma fonte de experiência pequena ou com ruídos estatísticos resulta em uma aproximação $\hat{T}$ aprendida pelo programa computacional que se adequa melhor do que $T$ aos dados da fonte de experiência utilizada mas que não generaliza corretamente os dados englobados em todo o domínio da função $T$\cite[p.79-80]{machine_learning}.

Neste trabalho, o programa de previsão de insuficiência cardíaca utilizou como fonte de experiência para seu treinamento um conjunto de dados\cite{larxel_dataset} composto de exemplos com variáveis e o resultado da aplicação dessas variáveis à função $T$, tomando parte, assim, na categoria de problemas de aprendizado supervisionado. O capítulo \ref{chap:procedimento_adotado} aborda alguns cuidados utilizados para evitar a ocorrência de sobreajuste no processo de aprendizado.

\subsection{Modelos de aprendizado}

Com a definição da tarefa a ser realizada e da fonte de experiência, devemos escolher um ou mais modelos de aprendizado de máquina que serão treinados pelo programa computacional. Cada modelo de aprendizado de máquina possui uma forma específica para representar a função $\hat{T}$ e para aprender com a fonte de experiência disponibilizada. Dessa forma, antes de qualquer outra consideração, o modelo de treinamento escolhido deve ser compatível com a categoria de aprendizado estabelecida pela fonte de experiência escolhida. Cada modelo de aprendizado também possui um conjunto de hiper-parâmetros que são utilizados para a realização de ajustes finos na representação de $\hat{T}$ utilizada pelo modelo e no método de aprendizado empregado. Com a otimização desses hiper-parâmetros, o modelo de aprendizado pode aprender mais com a fonte de experiência e, consequentemente, melhorar a aproximação $\hat{T}$ obtida em relação à função $T$.

A escolha do modelo de aprendizado apresenta um \textit{trade-off} importante na representação escolhida para a função $\hat{T}$: quanto maior a representatividade do modelo, maior é o número de dados necessários para treiná-lo\cite[p.8]{machine_learning} e menor é a interpretabilidade do modelo\cite[p.25]{statistical_learning}. Essa é uma consideração importante a ser feita dependendo do tamanho da fonte de experiência utilizada e do próposito para o qual o aprendizado de máquina está sendo empregado.

Neste trabalho, o programa de previsão de insuficiência cardíaca avaliou dois tipos de modelos de aprendizado distintos para aprendizado de máquina: perceptron multicamada e floresta aleatória. Ambos estes modelos são utilizados para problemas de aprendizado da categoria de aprendizado supervisionado e possuem um grande conjunto de combinações possíveis de hiper-parâmetros.

\subsubsection{Perceptron multicamada}

O modelo de aprendizado de perceptron multicamada é um tipo de rede neural artificial, classe de modelos de aprendizado inspirada pelos sistemas de aprendizado biológicos compostos por redes de neurônios interconectados. As redes neurais são compostas por várias unidades computacionais agrupadas em camadas, cada unidade computacional recebendo um conjunto de entradas reais e fornecendo uma saída real, a qual pode ser utilizada como entrada de outra unidade computacional da rede neural\cite[p.82]{machine_learning}.

O perceptron é uma das unidades computacionais que podem ser utilizadas para compor uma rede neural artificial. Cada perceptron pode ser descrito por uma função $P(\vec{x} \cdot \vec{w})$, onde $\vec{x} = (1, x_{1}, x_{2}, ..., x_{n}) \in \Re^{n+1}$ é um vetor de entradas reais, e $\vec{w} = (w_{0}, w_{1}, ..., w_{n}) \in \Re^{n+1}$ é um vetor de pesos, onde cada valor $w_{i}$ representa o peso que o perceptron fornece à entrada $x_{i}$. Existem várias funções que podem ser utilizadas como a função $P$ que descreve o perceptron com alguns dos exemplos mais comuns sendo as funções degrau, as funções sigmoide, as funções de Unidade Linear Retificada (ReLU)\nomenclature{ReLU}{Unidade Linear Retificada} e a função identidade. Os valores de cada peso $w_{i}$ do perceptron variam ao longo do processo de aprendizagem com a fonte de experiência e geralmente possuem como valor inicial um número aleatório pertencente ao intervalo $(-1, 1)$.

Quando temos uma rede neural composta por várias camadas de perceptrons, temos o modelo de aprendizado conhecido como perceptron multicamada. Ao longo do processo de aprendizado supervisionado do perceptron multicamada, as entradas de cada exemplo do conjunto de dados são inseridas na rede e a saída fornecida pela rede é comparada com a saída registrada no conjunto de dados. Após a execução de certo número de exemplos, o perceptron multicamada executa um algoritmo de retropropagação para ajustar os pesos de seus perceptrons de forma a minimizar uma medida de erro, que geralmente é o erro quadrático entre as saídas geradas pela rede neural e as saídas registradas no conjunto de dados\cite[p.97]{machine_learning}. Cada treinamento realizado desta maneira sobre todo o conjunto de dados é conhecido como uma época e, após certo número de épocas de aprendizado, o modelo de perceptron multicamada obtém uma boa aproximação $\hat{T}$. Para a execução do algoritmo de retropropagação, é necessário que a função $P$ utilizada pelos perceptrons de cada camada seja uma função diferenciável, de forma que todas as funções de ativação utilizadas em um perceptron multicamada devem ser diferenciáveis.

No modelo de perceptron multicamada, alguns dos hiper-parâmetros que podemos modificar, além das dimensões da rede neural em termos de número de camadas e de número de perceptrons por camada, incluem a função diferenciável utilizada como função de ativação $P$ dos perceptrons de uma camada, a medida de erro minimizada na execução do algoritmo de retropropagação, o número de épocas de aprendizado, a aplicação de um fator de \textit{dropout} no aprendizado de uma camada e a utilização de um fator de regularização sobre os pesos dos perceptrons de uma camada. Destes, os fatores de \textit{dropout} e de regularização de pesos merecem maior explicação. Um fator $d \in [0, 1)$ de \textit{dropout} para uma camada de perceptrons faz com que, durante o treinamento, os perceptrons dessa camada sobrescrevam aleatoriamente uma fração $d$ de suas entradas com o valor 0, permitindo que todas as entradas sejam treinadas para influenciarem a ativação do perceptron. Já o fator de regularização de pesos estabelece, para cada perceptron de uma camada, uma penalidade na medida de erro que é proporcional à uma função dos pesos do perceptron, como soma ou soma quadrática, incentivando os pesos do perceptron a se manterem baixos para diminuir a complexidade da aproximação $\hat{T}$ aprendida\cite[p.111]{machine_learning}.

\subsubsection{Floresta aleatória}

O modelo de aprendizado do tipo floresta aleatória é um tipo de modelo agrupado sob a classificação de modelos de aprendizado baseados em árvores de decisão, os quais utilizam uma ou mais árvores de decisão para gerar uma aproximação $\hat{T}$ para a função $T$. Dessa forma, antes de podermos explicar o modelo de aprendizado do tipo floresta aleatória, devemos explicar o modelo de aprendizado do tipo árvore de decisão.

Uma árvore de decisão é um modelo de aprendizado que busca subdividir o domínio $V$ da função $T$ em um número de regiões menores $R_{1}, R_{2}, ..., R_{n}$ de forma que $R_{i} \cap R_{j} = \emptyset \text{ } \forall i \ne j$ e que essas regiões agrupem os exemplos fornecidos pelo conjunto de dados utilizado como fonte de experiência. A representação final da função $\hat{T}$ é um conjunto de regras de divisão de $V$ cuja representação gráfica se assemelha a uma árvore e cujo resultado fornecido para um arranjo de variáveis que se encontre dentro de uma região $R_{i} \subset V$ equivale à média ou à moda dos resultados de todos os exemplos do conjunto de dados utilizado como fonte de experiência cujas variáveis também se encontrem dentro da região $R_{i}$\cite[p.303]{statistical_learning}. Cada subdivisão feita em $V$ para gerar uma nova região $R_{i} \subset V$ busca sempre diminuir ao máximo, por meio de uma abordagem gulosa, a soma dos quadrados dos erros entre os resultados da aplicação de $\hat{T}$ e os resultados dos exemplos do conjunto de dados usado como fonte de experiência, no caso de um problema de regressão, ou uma medida de variância dos agrupamentos de classes dos exemplos do conjunto de dados da fonte de experiência, no caso de um problema de classificação\cite[p.306-307, 312]{statistical_learning}.

A utilização de uma única árvore de decisão para gerar uma aproximação $\hat{T}$ para a função $T$ não fornece bons resultados, mas a utilização de uma aproximação $\hat{T}$ gerada pelo consenso do resultado de várias árvores de decisão pode fornecer resultados com excelente acurácia\cite[p.303]{statistical_learning}. O modelo de aprendizado de floresta aleatória gera uma aproximação $\hat{T}$ com base no consenso de um número arbitrário de árvores de decisão, onde cada árvore de decisão é gerada utilizando um conjunto de dados obtido pela escolha aleatória com possibilidade de repetição dos exemplos do conjunto de dados que serve como fonte de experiência, de forma que ambos conjuntos de dados tenham o mesmo tamanho (técnica conhecida como \textit{bootstrapping}). Além disso, na construção de uma floresta aleatória, sempre que uma das árvores de decisão subdivide o dominío $V$, esta considera apenas um subconjunto, também aleatório, de variáveis do conjunto de dados utilizado como fonte de experiência. Estas duas características auxiliam na diminuição da variância dos resultados gerados pelo consenso das árvores de decisão, aumentando a acurácia dos resultados da aproximação $\hat{T}$ fornecida por uma floresta aleatória\cite[p.316-321]{statistical_learning}.

No modelo de floresta aleatória, alguns dos hiper-parâmetros que podemos modificar incluem: o número de árvores de decisão utilizado; a medida de variância ou de erro a ser minimizada na realização das subdivisões do domínio $V$; o tamanho do subconjunto de variáveis, pertencentes ao conjunto de dados utilizado como fonte de experiência, consideradas na realização de cada subdivisão do domínio $V$; e vários hiper-parâmetros utilizados para limitar o crescimento de cada árvore de decisão como número mínimo de exemplos para compor uma folha, número mínimo de exemplos para realizar uma nova subdivisão do domínio $V$, profundidade máxima das árvores de decisão e número máximo de folhas das árvores de decisão. A utilização de hiper-parâmetros que limitam o crescimento das árvores de decisão resulta em uma aproximação $\hat{T}$ com menor complexidade, o que diminui a probabilidade de ocorrência de sobreajuste\cite[p.307]{statistical_learning}.

\subsection{Métricas de desempenho}

Por fim, devemos escolher uma ou mais métricas de desempenho para mensurar objetivamente o desempenho do programa de aprendizado de máquina no aprendizado da tarefa $T$ proposta. Cada categoria de aprendizado possui suas próprias métricas de desempenho e a natureza do problema proposto também influencia as métricas utilizadas. Por exemplo, para a categoria de aprendizado supervisionado, as métricas de desempenho utilizadas diferem em casos de problemas de regressão e de classificação. A escolha das métricas de desempenho é importante para que a avaliação dos modelos de aprendizado treinados ilustre adequadamente o desempenho do modelo em aprender a tarefa $T$ em conformidade com o objetivo a ser atingido pelo uso do aprendizado de máquina.

Neste trabalho, em que o problema estudado é categorizado como um problema de aprendizado supervisionado de classificação, a métrica de desempenho adotada foi a média da acurácia das classificações feitas por um modelo de treinamento para os exemplos contidos em um número de subconjuntos de dados utilizados para fins de validação ou de teste e, portanto, não utilizados no processo de aprendizado do modelo. Outras métricas que podem ser utilizadas para problemas dessa categoria incluem a precisão das classificações realizadas, a sensibilidade (\textit{recall}) das classificações realizadas e o Coeficiente de Correlação de Matthews (MCC)\nomenclature{MCC}{Coeficiente de Correlação de Matthews}.

Para definirmos matematicamente cada uma dessas métricas de problemas de aprendizado supervisionado de classificação, vamos estabelecer que $N$ é o número total de previsões feitas por um modelo, $T_{p}$ é o número de previsões que foram verdadeiros positivos, $T_{n}$ é o número de previsões que foram verdadeiros negativos, $F_{p}$ é o número de previsões que foram falsos positivos, e $F_{n}$ é o número de previsões que foram falsos negativos. Dessa forma, temos que $N = T_{p} + F_{p} + T_{n} + F_{n}$. Com essa definições, podemos definir a acurácia ($Acc$) pela equação \ref{eq:accuracy}, a precisão ($Precision$) pela equação \ref{eq:precision}, a sensibilidade ($Recall$) pela equação \ref{eq:recall}, e o Coeficiente de Correlação de Matthews ($MCC$) pela equação \ref{eq:mcc}.

\begin{equation} \label{eq:accuracy}
  Acc = \frac{T_{p} + T_{n}}{N}
\end{equation}

\begin{equation} \label{eq:precision}
  Precision = \frac{T_{p}}{T_{p} + F_{p}}
\end{equation}

\begin{equation} \label{eq:recall}
  Recall = \frac{T_{p}}{T_{p} + F_{n}}
\end{equation}

\begin{equation} \label{eq:mcc}
  MCC = \frac{(T_{p} \times T_{n}) - (F_{p} \times F_{n})}{ \\
    \sqrt{ \\
      (T_{p} + F_{p}) \times (T_{p} + F_{n}) \times (T_{n} + F_{p}) \\
      \times (T_{n} + F_{n}) \\
    } \\
  }
\end{equation}

\section{Funcionamento de um website}

O funcionamento de um website é um processo complexo que envolve vários aspectos relacionados ao servidor que hospeda o website, ao protocolo utilizado pelo navegador de internet do cliente para localizar esse servidor na internet, e ao processo de comunicação entre o servidor que hospeda o website e o navegador de internet para permitir que o cliente utilize o website. Para este trabalho, nosso foco estará apenas na estrutura do website em si, e não nos protocolos utilizados por um navegador de internet para localizar o website na internet ou em detalhes mais complexos da comunicação entre o navegador de internet do cliente e o servidor que hospeda o website. De forma bem resumida, podemos dividir a estrutura padrão de um website em dois componentes básicos, os quais não precisam ser implementados no mesmo programa ou mesmo armazenados no mesmo servidor: o \textit{backend} e o \textit{frontend}.

O \textit{backend} de um website engloba o banco de dados utilizado para o armazenamento dos dados do website e as funções ou métodos utilizados para acessar, criar, editar, apagar ou de alguma forma gerenciar esses dados. Dessa forma, o \textit{backend} geralmente é visualizado apenas pelos programadores envolvidos na construção do website e não pelos usuários e é utilizado para fornecer os dados que são visualizados e acessados pelos usuários no uso do website.

O \textit{frontend} de um website é composto pelas telas do website que são renderizadas pelo browser e por funções ou métodos utilizados para requisitar dados ao \textit{backend} do website e determinar quais telas ou elementos de telas que devem ser renderizados conforme as ações do usuário. Dessa forma, os usuários de um website geralmente interagem apenas com o \textit{frontend} do website ao utilizar um navegador de internet.

Geralmente, o \textit{backend} e o \textit{frontend} de um website são construídos no mesmo programa para serem executados de maneira acoplada no mesmo servidor e poderem se comunicar por meio de chamadas locais de funções. Entretanto, um novo padrão de projeto de websites tem se estabelecido, no qual o \textit{backend} do website é construído em um programa separado do \textit{frontend} para poder ser acessado diretamente por meio de requisições do Protocolo de Transferência de Hipertexto (HTTP)\nomenclature{HTTP}{Protocolo de Transferência de Hipertexto} ou até mesmo por mais de um \textit{frontend} simultaneamente. Quando isso ocorre, dizemos que o \textit{backend} do website se trata de uma Interface de Programação de Aplicações (\textit{API})\nomenclature{API}{Interface de Programação de Aplicações}.

A utilização de uma \textit{API} como \textit{backend} do website faz com que tenhamos que executar o \textit{backend} do website e o \textit{frontend} do website em servidores diferentes ou em portas diferentes do mesmo servidor, de forma que a comunicação entre \textit{backend} e \textit{frontend} do website passa a ser feita por requisições HTTP e não pela execução de funções locais. Embora essa arquitetura gere mais trabalho na programação da lógica do website, ela abre um número maior de possibilidades para o programador ao permitir que diversos \textit{frontends} no formato de interfaces web, aplicativos ou clientes de protocolo HTTP utilizem, simultaneamente, a API de \textit{backend} criada. Além disso, essa arquitetura facilita a realização de testes com o website ao permitir que a separação dos testes de \textit{backend} e de \textit{frontend} seja feita com mais facilidade.

No próximo capítulo, detalharemos o procedimento adotado ao longo deste trabalho.

	\chapter{Procedimento adotado} \label{procedimento_adotado}

Neste capítulo, o procedimento adotado ao longo do trabalho é detalhado com o intuito de permitir ao leitor compreender melhor a lógica por trás de algumas escolhas feitas no decorrer do trabalho e de explorar alguns detalhes importantes da implementação dos módulos de programação.

\section{Conjunto de dados}

Como descrito na introdução, o intuito deste trabalho foi elaborar um projeto que utilizasse aprendizado de máquinas e proporcionasse alguma utilidade prática. Para isso, foi necessária a escolha de um conjunto de dados que possibilitasse a criação de uma aplicação completa com base em aprendizado de máquinas em tempo útil para a realização deste trabalho. Tendo isso em mente, a decisão recaiu sobre um conjunto de dados\cite{larxel_dataset} originado de um artigo científico\cite{chicco2020} voltado para o estudo de aprendizado de máquinas para a previsão de ocorrência de insuficiência cardíaca.

O conjunto de dados escolhido\cite{larxel_dataset} contém 299 registros de pacientes, cada um consistindo em 11 parâmetros relacionados à saúde geral e cardíaca de um paciente, 1 parâmetro indicando o tempo de acompanhamento médico do paciente e 1 resultado indicando se o paciente veio à óbito durante o acompanhamento realizado. Dessa forma, trata-se de um conjunto de dados pequeno, de fácil compreensão, que pudesse ser explorado com um certo grau de profundidade em um curto espaço de tempo, e com uma clara utilidade prática de auxiliar a prevenção da ocorrência de insuficiência cardíaca em pacientes com base em uma previsão realizada por meio de aprendizado de máquinas.

Além das características úteis do conjunto de dados em si para a realização deste trabalho, o artigo científico que originou o conjunto de dados\cite{chicco2020} consiste em um ótimo estudo comparativo do desempenho de diferentes tipos de modelos de dados na previsão de insuficiência cardíaca com base no conjunto de dados em si, proporcionando uma referência útil para a realização de escolhas do tipo de modelo de aprendizado de máquinas a ser utilizado no conjunto de dados e para a comparação dos resultados atingidos no treinamento de modelos de aprendizado de máquinas.

Apesar dessas excelentes qualidades, devemos ressaltar que o conjunto de dados escolhido não é sem falhas. A baixa quantidade de registros no conjunto de dados faz com que o treinamento de modelos de aprendizado de máquinas seja extremamente suscetível à \textit{overfitting}, resultando em um modelo que não forneça bons resultados em aplicações práticas. Também podemos citar a baixa quantidade de registros em que o paciente veio à óbito, agravada pelas subdivisões do conjunto de dados em dados de treinamento, de teste e de validação. Essa ausência de robustez do conjunto de dados aumenta a probabilidade de que o modelo de aprendizado de máquinas não se familiarize o suficiente com situações indicativas de insuficiência cardíaca e gere falsos negativos, em que o sistema não prediz a insuficiência cardíaca e o paciente vem à óbito. A forma como essas limitações da base de dados foram tratadas será descrita posteriormente.

%label cria um rótulo para o objeto, para permitir que ele seja referenciado com o comando \ref{nome-do-rotulo}


% Neste capítulo é apresentada uma arquitetura para provimento de comércio eletrônico
% para o Sistema Brasileiro de TV Digital (SBTVD) \nomenclature{SBTVD}{Sistema Brasileiro de TV Digital}. A mesma é uma arquitetura distribuída, baseada em componentes
% reutilizáveis, os \textit{Web Services}, conhecida como Arquitetura Orientada a Serviços.
%
% Segundo \cite{soares2007ginga}:
%
% \begin{quote}
% 	Lorem ipsum dolor sit amet, consectetuer adipiscing elit. Ut purus elit, vesti- bulum ut, placerat ac, adipiscing vitae, felis. Curabitur dictum gravida mauris. Nam arcu libero, nonummy eget, consectetuer id, vulputate a, magna.
% \end{quote}
%
% Desta forma, a arquitetura proposta foi definida, incluindo a implementação de um \textit{framework} de comunicação (baseado nos protocolos HTTP e SOAP) que é apresentado sucintamente neste capítulo, e em mais detalhes no Capítulo \ref{capitulo2}. Mais detalhes podem ser consultados em \cite{soares2007ginga}. Veja um exemplo na Listagem \ref{list:server}, que foi adaptada de \url{http://manoelcampos.com}.
%
% Lorem ipsum dolor sit amet, consectetuer adipiscing elit. Ut purus elit, vestibulum ut, placerat ac, adipiscing vitae, felis. Curabitur dictum gravida mauris. Nam arcu libero, no- nummy eget, consectetuer id, vulputate a, magna. Donec vehicula augue eu neque. Pel- lentesque habitant morbi tristique senectus et netus et malesuada fames ac turpis egestas. Mauris ut leo. Cras viverra metus rhoncus sem. Nulla et lectus vestibulum urna fringilla ultrices. Phasellus eu tellus sit amet tortor gravida placerat. Integer sapien est, iaculis in, pretium quis, viverra ac, nunc. Praesent eget sem vel leo ultrices bibendum. Aenean fauci- bus. Morbi dolor nulla, malesuada eu, pulvinar at, mollis ac, nulla. Curabitur auctor semper nulla. Donec varius orci eget risus. Duis nibh mi, congue eu, accumsan eleifend, sagittis quis, diam. Duis eget orci sit amet orci dignissim rutrum.
%
% Nulla malesuada porttitor diam. Donec felis erat, congue non, volutpat at, tincidunt tristi- que, libero. Vivamus viverra fermentum felis. Donec nonummy pellentesque ante. Phasellus adipiscing semper elit. Proin fermentum massa ac quam. Sed diam turpis, molestie vitae, placerat a, molestie nec, leo. Maecenas lacinia. Nam ipsum ligula, eleifend at, accumsan nec, suscipit a, ipsum. Morbi blandit ligula feugiat magna. Nunc eleifend consequat lorem. Sed lacinia nulla vitae enim. Pellentesque tincidunt purus vel magna. Integer non enim. Praesent euismod nunc eu purus. Donec bibendum quam in tellus. Nullam cursus pulvinar lectus. Donec et mi. Nam vulputate metus eu enim. Vestibulum pellentesque felis eu massa.
%
% \lstset{caption=Exemplo de aplicação servidora, label=list:server}
% \begin{lstlisting}[language=C]
% int main()
% {
%     FILE *fp;     int len;
%     static const int SIZE = 1024;
%     struct sockaddr_in me, target;
%     int sock=socket(AF_INET,SOCK_DGRAM,0);
%     char arquivo[SIZE];
%     me.sin_family=AF_INET;
%     me.sin_addr.s_addr=htonl(INADDR_ANY); // endereco IP local
%     me.sin_port=htons(0); // porta local (0=auto assign)
%     bind(sock,(struct sockaddr *)&me,sizeof(me));
%     target.sin_family=AF_INET;
%     target.sin_addr.s_addr=inet_addr("192.168.68.217"); // host local
%     target.sin_port=htons(8450); // porta de destino
%
%     if ((fp = fopen("video1.mp4","rb")) == NULL){
%         printf("Arquivo nao pode ser aberto.\n"); return -1;
%     }
%
%     while(!feof(fp)) {
%         len = fread(arquivo, 1, sizeof(arquivo), fp);
%         sendto(sock,arquivo,sizeof(arquivo),0,(struct sockaddr *)&target,sizeof(target));
%     }
%     sendto(sock,"FIM",sizeof("FIM"),0,(struct sockaddr *)&target,sizeof(target));
%     close(sock);
%     return 0;
% }
% \end{lstlisting}

	\chapter{Resultados} \label{chap:resultados}

Neste capítulo, os resultados obtidos neste trabalho são abordados com o intuito de explicá-los ao leitor e de realizar uma breve análise englobando as expectativas iniciais do autor e possíveis consequências dos resultados obtidos.

\section{Avaliação de modelos de treinamento}

Esta seção tem como objetivos descrever brevemente as avaliações realizadas, identificar os melhores modelos de treinamento dentre os modelos avaliados e analisar os resultados obtidos pelos melhores modelos de treinamento. Acreditamos que, dessa forma, o leitor será primeiro provido das informações relevantes das avaliações realizadas para apenas então ser apresentado aos resultados e análises baseados nessas avaliações.

\subsection{Descrição das avaliações realizadas}

No total, foram realizadas 30 avaliações de modelos de treinamento de aprendizado de máquinas, 18 para modelos do tipo perceptron multicamada e 12 para modelos do tipo florestas randômicas. Estas avaliações permitiram que um total de 5346 modelos de treinamento fossem avaliados segundo o método de avaliação proposto, sendo 90 destes modelos do tipo perceptron multicamada e 5256 destes modelos do tipo florestas randômicas. Por fim, podemos observar que, dentre os modelos testados, um total de 18 modelos do tipo perceptron multicamada e 1053 modelos do tipo florestas randômicas foram classificados entre os melhores 20\% de todos os modelos de sua avaliação no quesito melhor média de acurácia para os dados de validação e, assim, foram testados em todos os 100 subconjuntos de teste disponíveis.

Podemos observar na tabela \ref{table:descricao_avaliacoes_perceptron_multicamada} uma breve descrição de cada avaliação realizada para modelos do tipo perceptron multicamada e na tabela \ref{table:descricao_avaliacoes_floresta_randomica} uma descrição análoga de cada avaliação realizada para modelos do tipo florestas randômicas. Estas descrições abordam os números das avaliações, o número de modelos avaliados, e as variações de hiperparâmetros empregadas. Para conveniência do leitor e concisão deste texto, as descrições de avaliações feitas para diferentes tipos de modelos de treinamento foram separadas em diferentes tabelas e algumas avaliações foram agrupadas na mesma descrição devido à sua semelhança nas variações de hiperparâmetros. As avaliações realizadas foram estruturadas de modo a testar uma grande quantidade de variações de hiperparâmetros para cada tipo de modelo com o intuito de determinar o conjunto de hiperparâmetros mais apto para uso em uma aplicação prática no website auxiliar. Em certos casos, os resultados de uma avaliação influenciaram diretamente as variações de hiperparâmetros das avaliações subsequentes.

\begin{table}[ht!]
  \begin{center}
  \setlength{\belowcaptionskip}{10pt}
  \footnotesize {
    \begin{tabular}{|p{1.5cm}|p{2cm}|p{12cm}|}
	  \hline
	  \textbf{Avalições} & \textbf{Número total de modelos} & \textbf{Variações de hiperparâmetros} \\
	  \hline
    03 & 5 & Perceptron multicamada com 1 camada de entrada, 1 camada densa e 1 camada de saída. Variação do tamanho da camada densa no intervalo $[100 .. 500]$ com deslocamentos de 100 em 100. Máximo de épocas de treinamento: 15 mil.\\
    \hline
    04, 05, 08, 09, 10 & 25 & Perceptron multicamada com 1 camada de entrada, 2 camadas densas e 1 camada de saída. Variação do tamanho da primeira camada densa no intervalo $[100 .. 500]$ com deslocamentos de 100 em 100 e do tamanho da segunda camada densa no intervalo $[20 .. 100]$ com deslocamentos de 20 em 20. Máximo de épocas de treinamento: 15 mil. \\
    \hline
    19 & 5 & Perceptron multicamada com 1 camada de entrada, 1 camada densa, 1 camada de \textit{dropout} com taxa 0.2 e 1 camada de saída. Variação do tamanho da camada densa no intervalo $[100 .. 500]$ com deslocamentos de 100 em 100. Máximo de épocas de treinamento: 20 mil. \\
    \hline
    20 - 24 & 25 & Perceptron multicamada com 1 camada de entrada, 2 camadas densas, 2 camadas de \textit{dropout} com taxa de \textit{dropout} 0.2 (intercaladas na sequência camada densa seguida de camada de \textit{dropout}) e 1 camada de saída. Variação do tamanho da primeira camada densa no intervalo $[100 .. 500]$ com deslocamentos de 100 em 100 e do tamanho da segunda camada densa no intervalo $[20 .. 100]$ com deslocamentos de 20 em 20. Máximo de épocas de treinamento: 20 mil. \\
    \hline
    25 & 5 & Perceptron multicamada com 1 camada de entrada, 1 camada densa com regularização de \textit{kernel} L2 com taxa 0.01 e 1 camada de saída. Variação do tamanho da camada densa no intervalo $[100 .. 500]$ com deslocamentos de 100 em 100. Máximo de épocas de treinamento: 20 mil. \\
    \hline
    26 - 30 & 25 & Perceptron multicamada com 1 camada de entrada, 2 camadas densas com regularização de \textit{kernel} L2 com taxa 0.01 e 1 camada de saída. Variação do tamanho da primeira camada densa no intervalo $[100 .. 500]$ com deslocamentos de 100 em 100 e do tamanho da segunda camada densa no intervalo $[20 .. 100]$ com deslocamentos de 20 em 20. Máximo de épocas de treinamento: 20 mil. \\
    \hline
    \end{tabular}
  }
  \caption{Breve descrição das avaliações realizadas para modelos de treinamento do tipo perceptron multicamada.}
  \label{table:descricao_avaliacoes_perceptron_multicamada}
  \end{center}
\end{table}

\begin{table}[ht!]
  \begin{center}
  \setlength{\belowcaptionskip}{10pt}
  \footnotesize {
    \begin{tabular}{|p{1.5cm}|p{2cm}|p{12cm}|}
	  \hline
	  \textbf{Avalições} & \textbf{Número total de modelos} & \textbf{Variações de hiperparâmetros} \\
	  \hline
	  01 & 30 & Variação do número de estimadores no intervalo $[10 .. 300]$ com deslocamentos de 10 em 10. Critério de divisão gini. Raiz quadrada do total de características como número máximo de características analisadas em uma divisão. \\
	  \hline
    02 & 60 & Variação do número de estimadores no intervalo $[190 .. 230]$ com deslocamentos de 5 em 5. Critérios de divisão: gini e entropia. Número máximo de características analisadas em uma divisão: raiz quadrada do total de características, logaritmo de base 2 do total de características e ausência de número máximo de características analisadas. \\
    \hline
    06 & 90 & 210 estimadores. Critério de divisão gini. Raiz quadrada do total de características como número máximo de características analisadas em uma divisão. Variação do mínimo de amostras para gerar uma folha no intervalo $[1 .. 10]$ e do mínimo de amostras para dividir um nó interno no intervalo $[2 .. 10]$. \\
    \hline
    07 & 450 & Variação do número de estimadores no intervalo $[200 .. 220]$ com deslocamentos de 5 em 5. Critérios de divisão: gini e entropia. Número máximo de características analisadas em uma divisão: raiz quadrada do total de características, logaritmo de base 2 do total de características e ausência de número máximo de características analisadas. Variação do mínimo de amostras para gerar uma folha no intervalo $[1 .. 3]$ e do mínimo de amostras para dividir um nó interno no intervalo $[5 .. 9]$. \\
    \hline
    11 & 66 & 210 estimadores. Critérios de divisão: gini e entropia. Número máximo de características analisadas em uma divisão: raiz quadrada do total de características, logaritmo de base 2 do total de características e ausência de número máximo de características analisadas. Variação da profundidade máxima das árvores no intervalo $[2 .. 12]$. \\
    \hline
    12, 13, 14 & 2250 & Variação do número de estimadores no intervalo $[200 .. 220]$ com deslocamentos de 5 em 5. Critérios de divisão: gini e entropia. Número máximo de características analisadas em uma divisão: raiz quadrada do total de características, logaritmo de base 2 do total de características e ausência de número máximo de características analisadas. Variação do mínimo de amostras para gerar uma folha no intervalo $[1 .. 3]$; do mínimo de amostras para dividir um nó interno no intervalo $[5 .. 9]$; e da profundidade máxima das árvores no intervalo $[2 .. 6]$. \\
    \hline
    15 & 66 & 210 estimadores. Critérios de divisão: gini e entropia. Número máximo de características analisadas em uma divisão: raiz quadrada do total de características, logaritmo de base 2 do total de características e ausência de número máximo de características analisadas. Variação do número máximo de folhas das árvores no intervalo $[4 .. 24]$ com deslocamentos de 2 em 2. \\
    \hline
    16, 17, 18 & 2250 & Variação do número de estimadores no intervalo $[200 .. 220]$ com deslocamentos de 5 em 5. Critérios de divisão: gini e entropia. Número máximo de características analisadas em uma divisão: raiz quadrada do total de características, logaritmo de base 2 do total de características e ausência de número máximo de características analisadas. Variação do mínimo de amostras para gerar uma folha no intervalo $[1 .. 3]$; do mínimo de amostras para dividir um nó interno no intervalo $[5 .. 9]$; e do número máximo de folhas das árvores no intervalo $[16 .. 20]$.  \\
    \hline
    \end{tabular}
  }
  \caption{Breve descrição das avaliações realizadas para modelos de treinamento do tipo floresta randômica.}
  \label{table:descricao_avaliacoes_floresta_randomica}
  \end{center}
\end{table}

Como já explicado no capítulo \ref{chap:procedimento_adotado}, a discrepância entre o número de modelos avaliados por avaliação e o número de avaliações em si para os modelos do tipo florestas randômicas e do tipo perceptron multicamada ocorre pois as avaliações feitas pelo autor com modelos do tipo perceptron multicamada foram consideravelmente lentas.

\subsection{Melhores resultados}

Dentre os 18 modelos do tipo perceptron multicamada e 1053 modelos do tipo florestas randômicas testados nos 100 subconjuntos de teste disponíveis, podemos observar os resultados dos 3 melhores modelos do tipo perceptron multicamada e dos 5 melhores modelos do tipo florestas randômicas na tabela \ref{table:ranking_melhores_modelos}, na qual os modelos foram rankeados pelo critério de melhor média de acurácia para os dados de teste. Com as informações desta tabela, poderemos realizar uma análise entre os diferentes tipos de modelos de treinamento e em relação à escolha de hiperparâmetros de cada tipo de modelo de treinamento.

\begin{table}[ht!]
  \begin{center}
  \setlength{\belowcaptionskip}{10pt}
  \footnotesize {
    \begin{tabular}{|p{1.25cm}|p{2cm}|p{2cm}|p{2.5cm}|p{6.75cm}|}
	  \hline
	  \textbf{Ranking} & \textbf{Tipo de \newline modelo} & \textbf{Identificação do modelo} & \textbf{Médias de \newline acurácia} & \textbf{Descrição do modelo} \\
	  \hline
    1º & Florestas randômica & Número da \newline avaliação: 18 \newline Número do \newline modelo: 21 & Teste: 0.760 \newline Validação: 0.762 & 215 estimadores. Critério de divisão gini. Ausência de número máximo de características analisadas em uma divisão. Mínimo de 3 amostras para gerar uma folha. Mínimo de 8 amostras para dividir um nó interno. Máximo de 19 folhas por árvore. \\
    \hline
    2º & Florestas randômica & Número da \newline avaliação: 18 \newline Número do \newline modelo: 58 & Teste: 0.760 \newline Validação: 0.759 & 220 estimadores. Critério de divisão gini. Ausência de número máximo de características analisadas em uma divisão. Mínimo de 2 amostras para gerar uma folha. Mínimo de 8 amostras para dividir um nó interno. Máximo de 19 folhas por árvore. \\
    \hline
    3º & Florestas randômica & Número da \newline avaliação: 18 \newline Número do \newline modelo: 25 & Teste: 0.759 \newline Validação: 0.761 & 210 estimadores. Critério de divisão gini. Ausência de número máximo de características analisadas em uma divisão. Mínimo de 3 amostras para gerar uma folha. Mínimo de 9 amostras para dividir um nó interno. Máximo de 20 folhas por árvore. \\
    \hline
    4º & Florestas randômica & Número da \newline avaliação: 18 \newline Número do \newline modelo: 27 & Teste: 0.759 \newline Validação: 0.761 & 205 estimadores. Critério de divisão gini. Ausência de número máximo de características analisadas em uma divisão. Mínimo de 3 amostras para gerar uma folha. Mínimo de 9 amostras para dividir um nó interno. Máximo de 20 folhas por árvore. \\
    \hline
    5º & Florestas randômica & Número da \newline avaliação: 18 \newline Número do \newline modelo: 49 & Teste: 0.759 \newline Validação: 0.760 & 200 estimadores. Critério de divisão gini. Ausência de número máximo de características analisadas em uma divisão. Mínimo de 3 amostras para gerar uma folha. Mínimo de 9 amostras para dividir um nó interno. Máximo de 17 folhas por árvore. \\
    \hline
    6º & Perceptron multicamada & Número da \newline avaliação: 09 \newline Número do \newline modelo: 00 & Teste: 0.707 \newline Validação: 0.694 & 1 camada de entrada, 1 camada densa de 400 unidades, 1 camada densa de 20 unidades e 1 camada de saída. Máximo de épocas de treinamento: 15 mil. \\
    \hline
    7º & Perceptron multicamada & Número da \newline avaliação: 24 \newline Número do \newline modelo: 00 & Teste: 0.704 \newline Validação: 0.716 & 1 camada de entrada, 1 camada densa de 500 unidades, 1 camada de \textit{dropout} com taxa de \textit{dropout} 0.2, 1 camada densa de 100 unidades, 1 camada de \textit{dropout} com taxa de \textit{dropout} 0.2 e 1 camada de saída. Máximo de épocas de treinamento: 20 mil. \\
    \hline
    8º & Perceptron multicamada & Número da \newline avaliação: 30 \newline Número do \newline modelo: 00 & Teste: 0.700 \newline Validação: 0.704 & 1 camada de entrada, 1 camada densa de 500 unidades com regularização de \textit{kernel} L2 com taxa 0.01, 1 camada densa de 100 unidades com regularização de \textit{kernel} L2 com taxa 0.01 e 1 camada de saída. Máximo de épocas de treinamento: 20 mil. \\
    \hline
    \end{tabular}
  }
  \caption{Ranking dos 3 melhores modelos do tipo perceptron multicamada e dos 5 melhores modelos do tipo florestas randômicas segundo a média de acurácia para dados de teste.}
  \label{table:ranking_melhores_modelos}
  \end{center}
\end{table}

Como observado na tabela \ref{table:ranking_melhores_modelos}, podemos notar que os modelos de florestas randômicas tiveram um desempenho significativamente superior em relação aos modelos do tipo perceptron multicamada. Embora isso possa ter sido exacerbado pela discrepância entre o número de modelos avaliados do tipo floresta randômica e do tipo perceptron multicamada, esse resultado já era esperado antes da realização das avaliações. Isso porque o artigo \cite{chicco2020} que analisa o conjunto de dados em questão \cite{larxel_dataset} também havia verificado um melhor desempenho dos modelos de treinamento do tipo florestas randômicas em relação aos modelos de treinamento do tipo perceptron multicamada, corroborando os resultados obtidos neste trabalho.

Analisando os hiperparâmetros dos modelos do tipo floresta randômica, podemos realizar algumas constatações. Primeiramente, podemos notar que o número de estimadores utilizados não foi uma característica que influenciou significativamente os resultados, pois todos os 5 valores de número de estimadores utilizados na avaliação 18 produziram um modelo no ranking de melhores modelos. Além disso, podemos também constatar que a utilização do coeficiente de gini se mostrou superior ao uso da entropia como critério de divisão. Também observamos que a ausência de um número máximo de características consideradas por divisão superou o uso de um número máximo de características consideradas por divisão sendo igual à raiz quadrada do total de características ou ao logaritmo de base 2 do total de características, o que contraria a teoria existente segundo a qual o uso da raiz quadrada do total de características como número máximo de características consideradas por divisão forneceria os melhores resultados \cite[p.319-321]{statistical_learning}. Outro aspecto que podemos observar é que a utilização de técnicas para limitar o número de divisões das árvores (mínimo de amostras para gerar uma folha, mínimo de amostras para dividir um nó interno) se mostraram úteis na obtenção dos melhores resultados, pois todos eles fizeram uso de ambas. Por fim, notamos que tanto a limitação no número de folhas como a limitação da profundidade máxima das árvores produziram bons resultados (embora esta última não tenha sido utilizada por nenhum dos melhores resultados, seu uso levou a um aumento da acurácia em relação a árvores que não a utilizaram). As duas últimas observações estão de acordo com a teoria por trás do modelo de árvores de decisão segundo a qual árvores complexas levam a \textit{overfitting} dos dados de treino, de forma que recursos para limitar a complexidade de uma árvore ou para extrair uma sub-árvore da árvore final podem diminuir o erro obtido sobre o subconjunto de dados de teste \cite[p.307-311]{statistical_learning}.

Analisando os hiperparâmetros dos modelos do tipo perceptron multicamada, também podemos realizar algumas constatações. Em primeiro lugar, notamos que todos os modelos de perceptrons multicamada presentes no ranking apresentam duas camadas densas, indicando que os modelos com duas camadas densas obtiveram melhor desempenho do que os modelos com apenas uma camada densa. Também é possível observar que os modelos com maior número de unidades nas camadas densas forneceram um melhor desempenho, em especial se combinados com um maior número de épocas de treinamento e uma camada de \textit{dropout} ou um \textit{kernel} de regularização aplicado aos pesos das camadas densas, tendo-se em vista que os 2 últimos modelos do tipo perceptron multicamada presentes no ranking possuem a maior quantidade de unidades para ambas as camadas densas dentre todos os modelos avaliados. O fato de o melhor modelo do tipo perceptron multicamada ter sido um modelo com camadas densas de tamanho 400 e 20, que não utiliza \textit{dropout} ou um \textit{kernel} de regularização aplicado aos pesos das camadas densas e que teve um limite máximo de 15 mil épocas de treinamento é um pouco surpreendente, em especial se considerarmos a média de acurácia para os dados de validação obtida por esse modelo. Com os resultados obtidos até aqui, creio que essa última constatação não possa ser explicada de maneira satisfatória, sendo necessária a realização de mais avaliações para se chegar a um resultado conclusivo.

Por último, os gráficos de resultados para os subconjuntos de validação e teste do melhor modelo no rankeamento foram disponibilizados na imagem \ref{fig:best_results_plot} para permitir ao leitor analisar com maior profundidade os resultados obtidos pelo modelo de treinamento que será utilizado para a previsão de dados no website.

\begin{figure}[h]
	\centering
  \subfigure[Gráfico de barras]{
    \centering
    \includegraphics[scale=0.5]{images/grafico_barras_resultados_melhor_modelo.png}
    \label{fig:best_results_barplot}
  }
  \subfigure[Gráfico de caixa]{
    \centering
    \includegraphics[scale=0.5]{images/grafico_caixa_resultados_melhor_modelo.png}
    \label{fig:best_results_boxplot}
  }
  \subfigure[Histograma]{
    \centering
    \includegraphics[scale=0.5]{images/histograma_resultados_melhor_modelo.png}
    \label{fig:best_results_histogram}
  }
	\caption{Gráficos de resultados para o melhor modelo do rankeamento.}
	\label{fig:best_results_plot}
\end{figure}


% \section{Uma associação}
%
% % Gera 3 parágrafos com texto aleatório, apenas para exemplo. Apague o comando para remover tal texto.
% \lipsum[1-3]
%
%
% \subsection{Motivos para abortar uma associação}
%
% %lista não numerada
% \begin{itemize}
% 	\item primeiro item;
% 	\item segundo item;
% 	\item terceiro item;
% 	\item quarto item.
% \end{itemize}
%
% \begin{figure}[h]
% 	\centering
% 	\includegraphics[scale=0.8]{images/red.png}
% 	\caption{Exemplo de imagem}
% 	\label{imagem1}
% \end{figure}
%
%
% Se durante o processo de configuração de uma associação for recebido como payload um hostname e esse hostname não puder ser resolvido em um tempo hábil deve se enviar um abort com a causa de erro de endereço não resolvido. Veja exemplo na Figura \ref{imagem1}.
%
%
% % Gera 4 parágrafos com texto aleatório, apenas para exemplo. Apague o comando para remover tal texto.
% \lipsum[1-4]
%
% A Tabela \ref{tabela1} a seguir apresenta os tipos de \textit{chunk} de um pacote do protocolo.
%
% \begin{table}[ht!]
%   \begin{center}
%   \setlength{\belowcaptionskip}{10pt} % espao entre caption e tabela
%   \footnotesize {
%       \begin{tabular}{|p{4cm}|p{9cm}|}
% 	  \hline
% 	  \textbf{Nome} & \textbf{Função} \\
% 	  \hline
% 	  Iniciar & Usado para iniciar uma associação \\
% 	  \hline
% 	  Confirmacao & Segunda mensagem de uma configuração de uma associação\\
% 	  \hline
% 	  Mensagem & Terceira mensagem de uma configuração de uma associação\\
% 	  \hline
% 	  Cookie & Quarta mensagem de uma configuração de uma associação\\
% 	  \hline
% 	  Dados & Dados da aplicação\\
% 	  \hline
%       \end{tabular}
%   }
%   \caption{Tipos de \textit{chunk} de um pacote SCTP}
%   \label{tabela1}
%   \end{center}
% \end{table}

	\chapter{Futuras pesquisas} \label{futuras_pesquisas}
% Como explicado no capítulo \ref{capitulo2}

	\chapter{Conclusão} \label{chap:conclusao}

Com base na introdução feita no capítulo \ref{chap:introducao} acerca dos objetivos deste trabalho e com base nos resultados deste trabalho apresentados no capítulo \ref{chap:resultados}, podemos concluir que este trabalho atingiu o objetivo proposto.

O estudo comparativo de modelos do tipo perceptron multicamada e floresta aleatória descrito nos capítulos \ref{chap:procedimento_adotado} e \ref{chap:resultados} foi capaz de enunciar claras distinções na acurácia de previsão do modelo de treinamento geradas pelo tipo de modelo de aprendizado de máquina utilizado e pelos hiperparâmetros escolhidos. A melhor média de acurácia de previsão obtida sobre os 100 subconjuntos de teste por um modelo do tipo floresta aleatória com base em um treinamento feito sobre 10 das 12 variáveis do conjunto de dados de insuficiência cardíaca utilizado\footnote{\cite{larxel_dataset}} foi de 76\%, superando a melhor marca de 74\% registrada no artigo de Chicco e Jurman\footnote{\cite{chicco2020}} para modelos do tipo floresta aleatória treinados com todas as 12 variáveis do conjunto de dados.

O website implementado funcionou adequadamente, conforme mencionado no capítulo \ref{chap:resultados}. Embora uma aplicação que oferece uma previsão de insuficiência cardíaca com acurácia de 76\% não possa ser disponibilizada ao público por razões éticas e legais relacionadas ao campo da medicina, podemos afirmar que o website implementado fornece um valor concreto aos seus usuários ao permitir que eles organizem as informações de seus pacientes e requisitem previsões de insuficiência cardíaca para estes.


	%\bibliographystyle{bibliography/abnt-num} % estilo bibliográfico ABNT numérico
	%\bibliographystyle{bibliography/abnt-alf} % estilo bibliográfico ABNT alfabético
	\bibliographystyle{bibliography/sbc}  % estilo bibliográfico da Sociedade Brasileira de Computação (SBC)

	%\renewcommand{\bibname}{REFERÊNCIAS BIBLIOGRÁFICAS} %Define o Caption da seção de bibliografia
	%\addcontentsline{toc}{chapter}{REFERÊNCIAS BIBLIOGRÁFICAS}

	%não pode ter espaço entre os nomes dos arquivos bib
	\bibliography{bibliography/referencias}

  % Seção comentada porque não foi utilizada.
	% \include{sections/apendice}
\end{document}
